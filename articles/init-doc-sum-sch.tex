\documentclass[12pt]{article}
%small modifications 9 May 07. DBS

\usepackage{amssymb}
\usepackage[spanish,activeacute]{babel}
\usepackage{enumerate}
\usepackage{array}
\usepackage{amsmath}
\usepackage{rotating}
\usepackage[latin1]{inputenc}



\newcommand{\CC}{\mathbb{C}}
\newcommand{\RR}{\mathbb{R}}
\newcommand{\ZZ}{\mathbb{Z}}
\newcommand{\NN}{\mathbb{N}}
\newcommand{\II}{\mathbb{I}}
\newcommand{\LL}{\mathbb{L}}
\newcommand{\HH}{\mathbb{H}}
\newcommand{\DD}{\mathbb{D}}
\newcommand{\mb}{\mathbf}
\newcommand{\ds}{\displaystyle}
\newcommand{\lm}{\lambda}
\newcommand{\wl}{\widehat{\lambda}}
\newcommand{\diag}{\mbox{diag}}



\setlength{\topmargin}{-10mm}
\setlength{\oddsidemargin}{-4mm}
\setlength{\evensidemargin}{0mm}
\setlength{\textheight}{239mm}
\setlength{\textwidth}{166mm}

\setlength{\parindent}{5mm}

%%% cambio a fonts sans serif
\renewcommand{\familydefault}{cmss}
\renewcommand{\bfdefault}{sbc}

%\pagestyle{empty}

\begin{document}
\begin{center}
{\Large \bf SIAG/LA INTERNATIONAL SUMMER SCHOOL ON NUMERICAL LINEAR ALGEBRA} \\[0.5cm]
{\large \bf INITIAL DOCUMENT}
\end{center}

\vspace{0.25cm}
\section{Presentation}

Prestigious summer schools regularly take place and are well established in many areas of knowledge. It is considered a standard activity for a student in a Ph.D. program in these areas to attend one of these schools at least once.
However these schools are not frequently organized in the mathematical community in general, and in the field of Numerical Linear Algebra in particular. Therefore, the {\em SIAM Activity Group on Linear Algebra} (SIAG/LA) undertakes the task of organizing a regular triennial {\em International Summer School on Numerical Linear Algebra} (ISSNLA) as one of its periodic activities. The ISSNLA will be held in different years than the SIAM Conference on Applied Linear Algebra and the Householder Meeting, with the exception of the first ISSNLA which will be organized in the year 2008. The next ISSNLA will be held in 2010, and, from then on, every three years. In addition, the ISSNLA will not coincide with the SIAM Annual Meeting, nor the ILAS Conference.

\section{Goals of the International Summer School}

The main goal of the SIAG/LA-ISSNLA is to offer accessible courses on current developments in Numerical Linear Algebra and on the applications of Numerical Linear Algebra to other disciplines. The courses will be taught by leading experts in the area, and these courses will be focused on recent research topics that have reached a significant maturity and whose impact is widely recognized by the community, but that are not usually included in text books, or in basic courses at the doctoral level.

\section{Target Audience}

The ISSNLA is mainly intended for doctoral students in any field where methods and algorithms of Numerical Linear Algebra are used. This implies that the courses will be self contained, their level will be accessible to a wide audience, and that they will be of interest to engineers, scientists, recent graduates, and persons working in industry who will benefit from an up-to-date view on the most modern techniques and algorithms in Numerical Linear Algebra. It is expected that course materials will be made available to the community via the SIAG/LA website.

\section{General Organization Issues}

%SIAG/LA will be responsible of organizing the ISSNLA every three years. 
SIAG/LA will be responsible for organizing the ISSNLA every three years. 
To this purpose some {\em partner institution or individuals} will be selected by the SIAG/LA Chair and the Program Director to find a place and funds to organize the ISSNLA. This partner institution or individuals will appoint a local organizing committee in consultation with SIAG/LA. The name of this partner institution will appear in the title of the course. The intention is that if a registration fee is established for the courses, that neither SIAM nor the partner institution or individuals organizing the school 
%might profit from the ISSNLA.
will profit from the ISSNLA.

In addition, two committees will be appointed by the SIAG/LA Chair and 
the Program Director:
%In the first place 
a {\em Steering Committee} with a few members that will decide on 
the subjects of the courses and the lecturers, and 
a wide {\em Advisory Committee} formed by leading experts in Numerical Linear Algebra and its Applications that will propose different subjects and lecturers for the courses to the Steering Committee, and offer other pertinent advice.

\section{Structure envisioned for the ISSNLA}

Although the specific format of the school may vary, the following recommendations are proposed: The ISSNLA will last five days. At least four different courses by four different lecturers will be presented. Each course will have 6-8 slots of 45 min. In addition, informal group sessions for problems, questions, computer 
labs, and discussions will be organized. There might be an occasional 
talk by a visiting lecturer.

\vspace{3cm}
\flushleft
{\em This document was written by Froil�n M. Dopico, Andreas Frommer, and Daniel B. Szyld in March, 2007.}
\end{document}
